\documentclass[a4paper,12pt]{article}

%%% HarrixLaTeXDocumentTemplate
%%% Версия 1.20
%%% Шаблон документов в LaTeX на русском языке. Данный шаблон применяется в проектах HarrixTestFunctions, MathHarrixLibrary, Standard-Genetic-Algorithm  и др.
%%% https://github.com/Harrix/HarrixLaTeXDocumentTemplate
%%% Шаблон распространяется по лицензии Apache License, Version 2.0.

%%% Поля и разметка страницы %%%
\usepackage{lscape} % Для включения альбомных страниц
\usepackage{geometry} % Для последующего задания полей

%%% Кодировки и шрифты %%%
\usepackage{pscyr} % Нормальные шрифты
\usepackage{cmap} % Улучшенный поиск русских слов в полученном pdf-файле
\usepackage[T2A]{fontenc} % Поддержка русских букв
\usepackage[utf8]{inputenc} % Кодировка utf8
\usepackage[english, russian]{babel} % Языки: русский, английский

%%% Математические пакеты %%%
\usepackage{amsthm,amsfonts,amsmath,amssymb,amscd} % Математические дополнения от AMS
% Для жиного курсива в формулах %
\usepackage{bm}
% Для рисования некоторых математических символов (например, закрашенных треугольников)
\usepackage{mathabx}

%%% Оформление абзацев %%%
\usepackage{indentfirst} % Красная строка
\usepackage{setspace} % Расстояние между строками
\usepackage{enumitem} % Для список обнуление расстояния до абзаца

%%% Цвета %%%
\usepackage[usenames]{color}
\usepackage{color}
\usepackage{colortbl}

%%% Таблицы %%%
\usepackage{longtable} % Длинные таблицы
\usepackage{multirow,makecell,array} % Улучшенное форматирование таблиц

%%% Общее форматирование
\usepackage[singlelinecheck=off,center]{caption} % Многострочные подписи
\usepackage{soul} % Поддержка переносоустойчивых подчёркиваний и зачёркиваний

%%% Библиография %%%
\usepackage{cite}

%%% Гиперссылки %%%
\usepackage{hyperref}

%%% Изображения %%%
\usepackage{graphicx} % Подключаем пакет работы с графикой
\usepackage{epstopdf}
\usepackage{subcaption}

%%% Отображение кода %%%
\usepackage{xcolor}
\usepackage{listings}
\usepackage{caption}

%%% Псевдокоды %%%
\usepackage{algorithm} 
\usepackage{algpseudocode}

%%% Рисование графиков %%%
\usepackage{pgfplots}

%%% HarrixLaTeXDocumentTemplate
%%% Версия 1.21
%%% Шаблон документов в LaTeX на русском языке. Данный шаблон применяется в проектах HarrixTestFunctions, MathHarrixLibrary, Standard-Genetic-Algorithm  и др.
%%% https://github.com/Harrix/HarrixLaTeXDocumentTemplate
%%% Шаблон распространяется по лицензии Apache License, Version 2.0.

%%% Макет страницы %%%
% Выставляем значения полей (ГОСТ 7.0.11-2011, 5.3.7)
\geometry{a4paper,top=2cm,bottom=2cm,left=2.5cm,right=1cm}

%%% Выравнивание и переносы %%%
\sloppy % Избавляемся от переполнений
\clubpenalty=10000 % Запрещаем разрыв страницы после первой строки абзаца
\widowpenalty=10000 % Запрещаем разрыв страницы после последней строки абзаца


%%% Библиография %%%
\makeatletter
\bibliographystyle{utf8gost71u}  % Оформляем библиографию по ГОСТ 7.1 (ГОСТ Р 7.0.11-2011, 5.6.7)
\renewcommand{\@biblabel}[1]{#1.} % Заменяем библиографию с квадратных скобок на точку
\makeatother

%%% Изображения %%%
\graphicspath{{images/}} % Пути к изображениям
% Поменять двоеточние на точку в подписях к рисунку
\RequirePackage{caption}
\DeclareCaptionLabelSeparator{defffis}{. }
\captionsetup{justification=centering,labelsep=defffis}

%%% Оглавление %%%
\renewcommand{\cftchapdotsep}{\cftdotsep}

%%% Абзацы %%%
% Отсупы между строками
\singlespacing
\setlength{\parskip}{0.3cm} % отступы между абзацами
\linespread{1.3} % Полуторный интвервал (ГОСТ Р 7.0.11-2011, 5.3.6)

% Оформление списков
\setlist{leftmargin=1.5cm,topsep=0pt}
% Используем дефис для ненумерованных списков (ГОСТ 2.105-95, 4.1.7)
\renewcommand{\labelitemi}{\normalfont\bfseries{--}}

%%% Цвета %%%
% Цвета для кода
\definecolor{string}{HTML}{B40000} % цвет строк в коде
\definecolor{comment}{HTML}{008000} % цвет комментариев в коде
\definecolor{keyword}{HTML}{1A00FF} % цвет ключевых слов в коде
\definecolor{morecomment}{HTML}{8000FF} % цвет include и других элементов в коде
\definecolor{сaptiontext}{HTML}{FFFFFF} % цвет текста заголовка в коде
\definecolor{сaptionbk}{HTML}{999999} % цвет фона заголовка в коде
\definecolor{bk}{HTML}{FFFFFF} % цвет фона в коде
\definecolor{frame}{HTML}{999999} % цвет рамки в коде
\definecolor{brackets}{HTML}{B40000} % цвет скобок в коде
% Цвета для гиперссылок
\definecolor{linkcolor}{HTML}{799B03} % цвет ссылок
\definecolor{urlcolor}{HTML}{799B03} % цвет гиперссылок
\definecolor{citecolor}{HTML}{799B03} % цвет гиперссылок
\definecolor{gray}{rgb}{0.4,0.4,0.4}
\definecolor{tableheadcolor}{HTML}{E5E5E5} % цвет шапки в таблицах
\definecolor{darkblue}{rgb}{0.0,0.0,0.6}
% Цвета для графиков
\definecolor{plotcoordinate}{HTML}{88969C}% цвет точек на координатых осях (минимум и максимум)
\definecolor{plotgrid}{HTML}{ECECEC} % цвет сетки
\definecolor{plotmain}{HTML}{97BBCD} % цвет основного графика
\definecolor{plotsecond}{HTML}{FF0000} % цвет второго графика, если графика только два
\definecolor{plotsecondgray}{HTML}{CCCCCC} % цвет второго графика, если графика только два. В сером виде.
\definecolor{darkgreen}{HTML}{799B03} % цвет темно-зеленого

%%% Отображение кода %%%
% Настройки отображения кода
\lstset{
language=C++, % Язык кода по умолчанию
morekeywords={*,...}, % если хотите добавить ключевые слова, то добавляйте
% Цвета
keywordstyle=\color{keyword}\ttfamily\bfseries,
%stringstyle=\color{string}\ttfamily,
stringstyle=\ttfamily\color{red!50!brown},
commentstyle=\color{comment}\ttfamily\itshape,
morecomment=[l][\color{morecomment}]{\#}, 
% Настройки отображения     
breaklines=true, % Перенос длинных строк
basicstyle=\ttfamily\footnotesize, % Шрифт для отображения кода
backgroundcolor=\color{bk}, % Цвет фона кода
frame=lrb,xleftmargin=\fboxsep,xrightmargin=-\fboxsep, % Рамка, подогнанная к заголовку
rulecolor=\color{frame}, % Цвет рамки
tabsize=3, % Размер табуляции в пробелах
% Настройка отображения номеров строк. Если не нужно, то удалите весь блок
%numbers=left, % Слева отображаются номера строк
%stepnumber=1, % Каждую строку нумеровать
%numbersep=5pt, % Отступ от кода 
%numberstyle=\small\color{black}, % Стиль написания номеров строк
% Для отображения русского языка
extendedchars=true,
literate={Ö}{{\"O}}1
  {Ä}{{\"A}}1
  {Ü}{{\"U}}1
  {ß}{{\ss}}1
  {ü}{{\"u}}1
  {ä}{{\"a}}1
  {ö}{{\"o}}1
  {~}{{\textasciitilde}}1
  {а}{{\selectfont\char224}}1
  {б}{{\selectfont\char225}}1
  {в}{{\selectfont\char226}}1
  {г}{{\selectfont\char227}}1
  {д}{{\selectfont\char228}}1
  {е}{{\selectfont\char229}}1
  {ё}{{\"e}}1
  {ж}{{\selectfont\char230}}1
  {з}{{\selectfont\char231}}1
  {и}{{\selectfont\char232}}1
  {й}{{\selectfont\char233}}1
  {к}{{\selectfont\char234}}1
  {л}{{\selectfont\char235}}1
  {м}{{\selectfont\char236}}1
  {н}{{\selectfont\char237}}1
  {о}{{\selectfont\char238}}1
  {п}{{\selectfont\char239}}1
  {р}{{\selectfont\char240}}1
  {с}{{\selectfont\char241}}1
  {т}{{\selectfont\char242}}1
  {у}{{\selectfont\char243}}1
  {ф}{{\selectfont\char244}}1
  {х}{{\selectfont\char245}}1
  {ц}{{\selectfont\char246}}1
  {ч}{{\selectfont\char247}}1
  {ш}{{\selectfont\char248}}1
  {щ}{{\selectfont\char249}}1
  {ъ}{{\selectfont\char250}}1
  {ы}{{\selectfont\char251}}1
  {ь}{{\selectfont\char252}}1
  {э}{{\selectfont\char253}}1
  {ю}{{\selectfont\char254}}1
  {я}{{\selectfont\char255}}1
  {А}{{\selectfont\char192}}1
  {Б}{{\selectfont\char193}}1
  {В}{{\selectfont\char194}}1
  {Г}{{\selectfont\char195}}1
  {Д}{{\selectfont\char196}}1
  {Е}{{\selectfont\char197}}1
  {Ё}{{\"E}}1
  {Ж}{{\selectfont\char198}}1
  {З}{{\selectfont\char199}}1
  {И}{{\selectfont\char200}}1
  {Й}{{\selectfont\char201}}1
  {К}{{\selectfont\char202}}1
  {Л}{{\selectfont\char203}}1
  {М}{{\selectfont\char204}}1
  {Н}{{\selectfont\char205}}1
  {О}{{\selectfont\char206}}1
  {П}{{\selectfont\char207}}1
  {Р}{{\selectfont\char208}}1
  {С}{{\selectfont\char209}}1
  {Т}{{\selectfont\char210}}1
  {У}{{\selectfont\char211}}1
  {Ф}{{\selectfont\char212}}1
  {Х}{{\selectfont\char213}}1
  {Ц}{{\selectfont\char214}}1
  {Ч}{{\selectfont\char215}}1
  {Ш}{{\selectfont\char216}}1
  {Щ}{{\selectfont\char217}}1
  {Ъ}{{\selectfont\char218}}1
  {Ы}{{\selectfont\char219}}1
  {Ь}{{\selectfont\char220}}1
  {Э}{{\selectfont\char221}}1
  {Ю}{{\selectfont\char222}}1
  {Я}{{\selectfont\char223}}1
  {і}{{\selectfont\char105}}1
  {ї}{{\selectfont\char168}}1
  {є}{{\selectfont\char185}}1
  {ґ}{{\selectfont\char160}}1
  {І}{{\selectfont\char73}}1
  {Ї}{{\selectfont\char136}}1
  {Є}{{\selectfont\char153}}1
  {Ґ}{{\selectfont\char128}}1
  {\{}{{{\color{brackets}\{}}}1 % Цвет скобок {
  {\}}{{{\color{brackets}\}}}}1 % Цвет скобок }
}
% Для настройки заголовка кода
\DeclareCaptionFont{white}{\color{сaptiontext}}
\DeclareCaptionFormat{listing}{\parbox{\linewidth}{\colorbox{сaptionbk}{\parbox{\linewidth}{#1#2#3}}\vskip-4pt}}
\captionsetup[lstlisting]{format=listing,labelfont=white,textfont=white}
\renewcommand{\lstlistingname}{Код} % Переименование Listings в нужное именование структуры
% Для отображения кода формата xml
\lstdefinelanguage{XML}
{
  morestring=[s]{"}{"},
  morecomment=[s]{?}{?},
  morecomment=[s]{!--}{--},
  commentstyle=\color{comment},
  moredelim=[s][\color{black}]{>}{<},
  moredelim=[s][\color{red}]{\ }{=},
  stringstyle=\color{string},
  identifierstyle=\color{keyword}
}

%%% Гиперссылки %%%
\hypersetup{pdfstartview=FitH,  linkcolor=linkcolor,urlcolor=urlcolor,citecolor=citecolor, colorlinks=true}

%%% Псевдокоды %%%
% Добавляем свои блоки
\makeatletter
\algblock[ALGORITHMBLOCK]{BeginAlgorithm}{EndAlgorithm}
\algblock[BLOCK]{BeginBlock}{EndBlock}
\makeatother

% Нумерация блоков
\usepackage{caption}% http://ctan.org/pkg/caption
\captionsetup[ruled]{labelsep=period}
\makeatletter
\@addtoreset{algorithm}{chapter}% algorithm counter resets every chapter
\makeatother
\renewcommand{\thealgorithm}{\thechapter.\arabic{algorithm}}% Algorithm # is <chapter>.<algorithm>

%%% Формулы %%%
%Дублирование символа при переносе
\newcommand{\hmm}[1]{#1\nobreak\discretionary{}{\hbox{\ensuremath{#1}}}{}}

%%% Таблицы %%%
% Раздвигаем в таблице без границ отступы между строками в новой команде
\newenvironment{tabularwide}%
{\setlength{\extrarowheight}{0.3cm}\begin{tabular}{  p{\dimexpr 0.45\linewidth-2\tabcolsep} p{\dimexpr 0.55\linewidth-2\tabcolsep}  }}  {\end{tabular}}
\newenvironment{tabularwide08}%
{\setlength{\extrarowheight}{0.3cm}\begin{tabular}{  p{\dimexpr 0.8\linewidth-2\tabcolsep} p{\dimexpr 0.2\linewidth-2\tabcolsep}  }}  {\end{tabular}}

% Многострочная ячейка в таблице
\newcommand{\specialcell}[2][c]{%
  {\renewcommand{\arraystretch}{1}\begin{tabular}[t]{@{}l@{}}#2\end{tabular}}}

% Многострочная ячейка, где текст не может выйти за границы
\newcolumntype{P}[1]{>{\raggedright\arraybackslash}p{#1}}
\newcommand{\specialcelltwoin}[2][c]{%
  {\renewcommand{\arraystretch}{1}\begin{tabular}[t]{@{}P{1.98in}@{}}#2\end{tabular}}}
  
% Команда для переворачивания текста в ячейке таблицы на 90 градусов
\newcommand*\rot{\rotatebox{90}}

%%% Рисование графиков %%%
\pgfplotsset{
every axis legend/.append style={at={(0.5,-0.13)},anchor=north,legend cell align=left},
tick label style={font=\tiny\scriptsize},
label style={font=\scriptsize},
legend style={font=\scriptsize},
grid=both,
minor tick num=2,
major grid style={plotgrid},
minor grid style={plotgrid},
axis lines=left,
legend style={draw=none},
/pgf/number format/.cd,
1000 sep={}
}
% Карта цвета для трехмерных графиков в стиле графиков Mathcad
\pgfplotsset{
/pgfplots/colormap={mathcad}{rgb255(0cm)=(76,0,128) rgb255(2cm)=(0,14,147) rgb255(4cm)=(0,173,171) rgb255(6cm)=(32,205,0) rgb255(8cm)=(229,222,0) rgb255(10cm)=(255,13,0)}
}
% Карта цвета для трехмерных графиков в стиле графиков Matlab
\pgfplotsset{
/pgfplots/colormap={matlab}{rgb255(0cm)=(0,0,128) rgb255(1cm)=(0,0,255) rgb255(3cm)=(0,255,255) rgb255(5cm)=(255,255,0) rgb255(7cm)=(255,0,0) rgb255(8cm)=(128,0,0)}
}

%%% Разное %%%
% Галочки для отмечания в тескте вариантов как OK
\def\checkmark{\tikz\fill[black,scale=0.3](0,.35) -- (.25,0) -- (1,.7) -- (.25,.15) -- cycle;}
\def\checkmarkgreen{\tikz\fill[darkgreen,scale=0.3](0,.35) -- (.25,0) -- (1,.7) -- (.25,.15) -- cycle;} 
\def\checkmarkred{\tikz\fill[red,scale=0.3](0,.35) -- (.25,0) -- (1,.7) -- (.25,.15) -- cycle;}
\def\checkmarkbig{\tikz\fill[black,scale=0.5](0,.35) -- (.25,0) -- (1,.7) -- (.25,.15) -- cycle;}
\def\checkmarkbiggreen{\tikz\fill[darkgreen,scale=0.5](0,.35) -- (.25,0) -- (1,.7) -- (.25,.15) -- cycle;} 
\def\checkmarkbigred{\tikz\fill[red,scale=0.5](0,.35) -- (.25,0) -- (1,.7) -- (.25,.15) -- cycle;}

%% Следующие блоки расскоментировать при необходимости

%%% Кодировки и шрифты %%%
%\ifxetex
%\setmainlanguage{russian}
%\setotherlanguage{english}
%\defaultfontfeatures{Ligatures=TeX,Mapping=tex-text}
%\setmainfont{Times New Roman}
%\newfontfamily\cyrillicfont{Times New Roman}
%\setsansfont{Arial}
%\newfontfamily\cyrillicfontsf{Arial}
%\setmonofont{Courier New}
%\newfontfamily\cyrillicfonttt{Courier New}
%\else
%\IfFileExists{pscyr.sty}{\renewcommand{\rmdefault}{ftm}}{}
%\fi

%%% Колонтитулы %%%
% Порядковый номер страницы печатают на середине верхнего поля страницы (ГОСТ Р 7.0.11-2011, 5.3.8)
%\makeatletter
%\let\ps@plain\ps@fancy              % Подчиняем первые страницы каждой главы общим правилам
%\makeatother
%\pagestyle{fancy}                   % Меняем стиль оформления страниц
%\fancyhf{}                          % Очищаем текущие значения
%\fancyhead[C]{\thepage}             % Печатаем номер страницы на середине верхнего поля
%\renewcommand{\headrulewidth}{0pt}  % Убираем разделительную линию

\title{HarrixQtLibraryForQWebView v.1.15}
\author{А.\,Б. Сергиенко}
\date{\today}


\begin{document}

%%% HarrixLaTeXDocumentTemplate
%%% Версия 1.20
%%% Шаблон документов в LaTeX на русском языке. Данный шаблон применяется в проектах HarrixTestFunctions, MathHarrixLibrary, Standard-Genetic-Algorithm  и др.
%%% https://github.com/Harrix/HarrixLaTeXDocumentTemplate
%%% Шаблон распространяется по лицензии Apache License, Version 2.0.

%%% Именования %%%
\renewcommand{\abstractname}{Аннотация}
\renewcommand{\alsoname}{см. также}
\renewcommand{\appendixname}{Приложение}
\renewcommand{\bibname}{Литература}
\renewcommand{\ccname}{исх.}
\renewcommand{\chaptername}{Глава}
%\renewcommand{\contentsname}{Содержание}
\renewcommand{\enclname}{вкл.}
\renewcommand{\figurename}{Рисунок}
\renewcommand{\headtoname}{вх.}
\renewcommand{\indexname}{Предметный указатель}
\renewcommand{\listfigurename}{Список рисунков}
\renewcommand{\listtablename}{Список таблиц}
\renewcommand{\pagename}{Стр.}
\renewcommand{\partname}{Часть}
\renewcommand{\refname}{Список литературы}
\renewcommand{\seename}{см.}
\renewcommand{\tablename}{Таблица}

%%% Псевдокоды %%%
% Перевод данных об алгоритмах
\renewcommand{\listalgorithmname}{Список алгоритмов}
\floatname{algorithm}{Алгоритм}

% Перевод команд псевдокода
\algrenewcommand\algorithmicwhile{\textbf{До тех пока}}
\algrenewcommand\algorithmicdo{\textbf{выполнять}}
\algrenewcommand\algorithmicrepeat{\textbf{Повторять}}
\algrenewcommand\algorithmicuntil{\textbf{Пока выполняется}}
\algrenewcommand\algorithmicend{\textbf{Конец}}
\algrenewcommand\algorithmicif{\textbf{Если}}
\algrenewcommand\algorithmicelse{\textbf{иначе}}
\algrenewcommand\algorithmicthen{\textbf{тогда}}
\algrenewcommand\algorithmicfor{\textbf{Цикл. }}
\algrenewcommand\algorithmicforall{\textbf{Выполнить для всех}}
\algrenewcommand\algorithmicfunction{\textbf{Функция}}
\algrenewcommand\algorithmicprocedure{\textbf{Процедура}}
\algrenewcommand\algorithmicloop{\textbf{Зациклить}}
\algrenewcommand\algorithmicrequire{\textbf{Условия:}}
\algrenewcommand\algorithmicensure{\textbf{Обеспечивающие условия:}}
\algrenewcommand\algorithmicreturn{\textbf{Возвратить}}
\algrenewtext{EndWhile}{\textbf{Конец цикла}}
\algrenewtext{EndLoop}{\textbf{Конец зацикливания}}
\algrenewtext{EndFor}{\textbf{Конец цикла}}
\algrenewtext{EndFunction}{\textbf{Конец функции}}
\algrenewtext{EndProcedure}{\textbf{Конец процедуры}}
\algrenewtext{EndIf}{\textbf{Конец условия}}
\algrenewtext{EndFor}{\textbf{Конец цикла}}
\algrenewtext{BeginAlgorithm}{\textbf{Начало алгоритма}}
\algrenewtext{EndAlgorithm}{\textbf{Конец алгоритма}}
\algrenewtext{BeginBlock}{\textbf{Начало блока. }}
\algrenewtext{EndBlock}{\textbf{Конец блока}}
\algrenewtext{ElsIf}{\textbf{иначе если }}

\maketitle

\begin{abstract}
Библиотека HarrixQtLibraryForQWebView --- библиотека для отображения различных данных в QWebView, включая графики.
\end{abstract}

\tableofcontents

\newpage

\section{Введение}

Библиотека HarrixQtLibraryForQWebView --- это библиотека для отображения различных данных в QWebView, включая графики..

Последнюю версию документа можно найти по адресу:

\href{https://github.com/Harrix/HarrixQtLibraryForQWebView}{https://github.com/Harrix/HarrixQtLibraryForQWebView}

Об установке библиотеки можно прочитать тут:

\href{http://blog.harrix.org/?p=1196}{http://blog.harrix.org/?p=1196}

С автором можно связаться по адресу \href{mailto:sergienkoanton@mail.ru}{sergienkoanton@mail.ru} или  \href{http://vk.com/harrix}{http://vk.com/harrix}.

Сайт автора, где публикуются последние новости: \href{http://blog.harrix.org/}{http://blog.harrix.org/}, а проекты располагаются по адресу \href{http://harrix.org/}{http://harrix.org/}.

\newpage
\section{Список функций}\label{section_listfunctions}
\textbf{Главные загрузочные функции}
\begin{enumerate}

\item \textbf{\hyperref[HQt_AddHtml]{HQt\_AddHtml}} --- Функция добавляет код html к существующему и сохраняет его в temp.html.

\item \textbf{\hyperref[HQt_BeginHtml]{HQt\_BeginHtml}} --- Функция обнуляет переменную HTML. Требуется когда нужно перезапустить показ информации в QWebView.

\end{enumerate}

\textbf{Показ графиков}
\begin{enumerate}

\item \textbf{\hyperref[HQt_DrawLine]{HQt\_DrawLine}} --- Функция возвращает строку с HTML кодом отрисовки линии по функции Function. Для добавление в html файл.

\item \textbf{\hyperref[THQt_ShowChartOfLine]{THQt\_ShowChartOfLine}} --- Функция возвращает строку с выводом некоторого графика по точкам с HTML кодами. Для добавление в html файл.

\item \textbf{\hyperref[THQt_ShowChartsOfLineFromMatrix]{THQt\_ShowChartsOfLineFromMatrix}} --- Функция возвращает строку с выводом графиков из матрицы по точкам с HTML кодами. Для добавление в html файл.

\item \textbf{\hyperref[THQt_ShowIndependentChartsOfLineFromMatrix]{THQt\_ShowIndependentChartsOfLineFromMatrix}} --- Функция возвращает строку с выводом графиков из матрицы по точкам с HTML кодами. Для добавление в html файл.

\item \textbf{\hyperref[THQt_ShowTwoChartsOfLine]{THQt\_ShowTwoChartsOfLine}} --- Функция возвращает строку с выводом некоторого двух графиков по точкам с HTML кодами. Для добавление в html файл. У обоих графиков одинаковый массив значений X.

\item \textbf{\hyperref[THQt_ShowTwoIndependentChartsOfLine]{THQt\_ShowTwoIndependentChartsOfLine}} --- Функция возвращает строку с выводом некоторого двух независимых графиков по точкам с HTML кодами. Для добавление в html файл.

\item \textbf{\hyperref[THQt_ShowTwoIndependentChartsOfPointsAndLine]{THQt\_ShowTwoIndependentChartsOfPointsAndLine}} --- Функция возвращает строку с выводом некоторого двух независимых графиков по точкам с HTML кодами. Для добавление в html файл. Один график выводится в виде точек, а второй в виде линии. Удобно для отображения регрессий. У обоих графиков разные массивы значений X и Y.

\end{enumerate}

\textbf{Показ математических выражений}
\begin{enumerate}

\item \textbf{\hyperref[THQt_ShowMatrix]{THQt\_ShowMatrix}} --- Функция возвращает строку с выводом некоторой матрицы VMHL\_Matrix с HTML кодами. Для добавление в html файл.

\item \textbf{\hyperref[THQt_ShowMatrix2]{THQt\_ShowMatrix2}} --- Функция возвращает строку с выводом некоторой матрицы VMHL\_Matrix с HTML кодами. Для добавление в html файл. В качестве матрицы выступает массив QStringList, где количество QStringList - это количество строк. Каждый QStringList --- это одна строка.

\item \textbf{\hyperref[THQt_ShowNumber]{THQt\_ShowNumber}} --- Функция возвращает строку с выводом некоторого числа VMHL\_X с HTML кодами. Для добавление в html файл.

\item \textbf{\hyperref[THQt_ShowVector]{THQt\_ShowVector}} --- Функция возвращает строку с выводом некоторого вектора VMHL\_Vector с HTML кодами. Для добавление в html файл.

\item \textbf{\hyperref[THQt_ShowVector2]{THQt\_ShowVector2}} --- Функция возвращает строку с выводом некоторого списка строк VMHL\_Vector с HTML кодами. Для добавление в html файл.

\item \textbf{\hyperref[THQt_ShowVectorT]{THQt\_ShowVectorT}} --- Функция возвращает строку с выводом некоторого вектора VMHL\_Vector в траснпонированном виде с HTML кодами. Для добавление в html файл.

\end{enumerate}

\textbf{Показ текста}
\begin{enumerate}

\item \textbf{\hyperref[HQt_ShowAlert]{HQt\_ShowAlert}} --- Функция возвращает строку с выводом некоторого предупреждения. Для добавление в html файл.

\item \textbf{\hyperref[HQt_ShowH1]{HQt\_ShowH1}} --- Функция возвращает строку с выводом некоторой строки в виде заголовка. Для добавление в html файл.

\item \textbf{\hyperref[HQt_ShowHr]{HQt\_ShowHr}} --- Функция возвращает строку с выводом горизонтальной линии. Для добавление в html файл.

\item \textbf{\hyperref[HQt_ShowSimpleText]{HQt\_ShowSimpleText}} --- Функция возвращает строку с выводом некоторой строки с HTML кодами без всякого излишества. Для добавление в html файл.

\item \textbf{\hyperref[HQt_ShowText]{HQt\_ShowText}} --- Функция возвращает строку с выводом некоторой строки с HTML кодами. Для добавление в html файл.

\item \textbf{\hyperref[THQt_NumberToText]{THQt\_NumberToText}} --- Функция выводит число VMHL\_X в строку.

\end{enumerate}


\newpage
\section{Функции}
\subsection{Главные загрузочные функции}

\subsubsection{HQt\_AddHtml}\label{HQt_AddHtml}

Функция добавляет код html к существующему и сохраняет его в temp.html.


\begin{lstlisting}[label=code_syntax_HQt_AddHtml,caption=Синтаксис]
void HQt_AddHtml(QString Html);
\end{lstlisting}

\textbf{Входные параметры:}

Html --- добавляемый текст.

\textbf{Возвращаемое значение:}

Отсутствует.


\subsubsection{HQt\_BeginHtml}\label{HQt_BeginHtml}

Функция обнуляет переменную HTML. Требуется когда нужно перезапустить показ информации в QWebView.


\begin{lstlisting}[label=code_syntax_HQt_BeginHtml,caption=Синтаксис]
void HQt_BeginHtml(QString Path);
\end{lstlisting}

\textbf{Входные параметры:}

Path --- путь к папке, в которой надо будет сохранять html код. В этой папке должен содержаться файл index.html.

\textbf{Возвращаемое значение:}

Отсутствует.


\subsection{Показ графиков}

\subsubsection{HQt\_DrawLine}\label{HQt_DrawLine}

Функция возвращает строку с HTML кодом отрисовки линии по функции Function. Для добавление в html файл.


\begin{lstlisting}[label=code_syntax_HQt_DrawLine,caption=Синтаксис]
QString HQt_DrawLine (double Left, double Right, double h, double (*Function)(double), QString TitleChart, QString NameVectorX, QString NameVectorY, QString NameLine, bool ShowLine, bool ShowPoints, bool ShowArea, bool ShowSpecPoints, bool RedLine);
QString HQt_DrawLine (double Left, double Right, double h, double (*Function)(double), QString TitleChart, QString NameVectorX, QString NameVectorY, bool ShowLine, bool ShowPoints, bool ShowArea, bool ShowSpecPoints, bool RedLine);
QString HQt_DrawLine (double Left, double Right, double h, double (*Function)(double), QString TitleChart, QString NameVectorX, QString NameVectorY, QString NameLine);
QString HQt_DrawLine (double Left, double Right, double h, double (*Function)(double));
\end{lstlisting}

\textbf{Входные параметры:}
 
    Left --- левая граница области;
 
    Right --- правая граница области;
 
    h --- шаг, с которым надо рисовать график;
 
    Function --- указатель на вычисляемую функцию;
 
    TitleChart --- заголовок графика;
 
    NameVectorX --- название оси Ox;
 
    NameVectorY --- название оси Oy;
 
    NameLine --- название первого графика (для легенды);
 
    ShowLine --- показывать ли линию;
 
    ShowPoints --- показывать ли точки;
 
    ShowArea --- показывать ли закрашенную область под кривой;
 
    ShowSpecPoints --- показывать ли специальные точки;
 
    RedLine --- рисовать ли красную линию, или синюю.

\textbf{Возвращаемое значение:}

Строка с HTML кодом.


\subsubsection{THQt\_ShowChartOfLine}\label{THQt_ShowChartOfLine}

Функция возвращает строку с выводом некоторого графика по точкам с HTML кодами. Для добавление в html файл.


\begin{lstlisting}[label=code_syntax_THQt_ShowChartOfLine,caption=Синтаксис]
template <class T> QString THQt_ShowChartOfLine (T *VMHL_VectorX,T *VMHL_VectorY, int VMHL_N, QString TitleChart, QString NameVectorX, QString NameVectorY, QString NameLine, bool ShowLine, bool ShowPoints, bool ShowArea, bool ShowSpecPoints, bool RedLine);
template <class T> QString THQt_ShowChartOfLine (T *VMHL_VectorX,T *VMHL_VectorY, int VMHL_N, QString TitleChart, QString NameVectorX, QString NameVectorY, bool ShowLine, bool ShowPoints, bool ShowArea, bool ShowSpecPoints, bool RedLine);
template <class T> QString THQt_ShowChartOfLine (T *VMHL_VectorX,T *VMHL_VectorY, int VMHL_N, QString TitleChart, QString NameVectorX, QString NameVectorY, QString NameLine);
template <class T> QString THQt_ShowChartOfLine (T *VMHL_VectorX,T *VMHL_VectorY, int VMHL_N);
\end{lstlisting}

\textbf{Входные параметры:}
 
    VMHL\_VectorX --- указатель на вектор координат X точек;
 
    VMHL\_VectorY --- указатель на вектор координат Y точек;
 
    VMHL\_N --- количество точек;
 
    TitleChart --- заголовок графика;
 
    NameVectorX --- название оси Ox;
 
    NameVectorY --- название оси Oy;
 
    NameLine --- название первого графика (для легенды);
 
    ShowLine --- показывать ли линию;
 
    ShowPoints --- показывать ли точки;
 
    ShowArea --- показывать ли закрашенную область под кривой;
 
    ShowSpecPoints --- показывать ли специальные точки;
 
    RedLine --- рисовать ли красную линию, или синюю.

\textbf{Возвращаемое значение:}

Строка с HTML кодами с выводимым графиком.

\textbf{Примечание:}

Используются случайные числа, так что рекомендуется вызвать в программе иницилизатор случайных чисел qsrand. Рекомендую так: qsrand(QDateTime::currentMSecsSinceEpoch () % 1000000);

Требует наличия в папке с html файлом файлы jsxgraph.css и jsxgraphcore.js из библиотеки JSXGraph.


\begin{lstlisting}[label=code_use_THQt_ShowChartOfLine,caption=Пример использования]
QString DS=QDir::separator();
QString path=QGuiApplication::applicationDirPath()+DS;//путь к папке

QString Html;
Html=HQt_BeginHtml ();

int N=6;
double *dataX=new double [N];
double *dataY=new double [N];
dataX[0]=7;dataY[0]=6;
dataX[1]=8;dataY[1]=4;
dataX[2]=10;dataY[2]=7;
dataX[3]=5;dataY[3]=12;
dataX[4]=14;dataY[4]=4;
dataX[5]=13;dataY[5]=8;

Html += THQt_ShowChartOfLine (dataX,dataY,N,"Тестовый график","x","y","линия",true,true,true,true,false);

delete []dataX;
delete []dataY;

Html+=HQt_EndHtml();
HQt_SaveFile(Html, path+"temp.html");
ui->webView->setUrl(QUrl::fromLocalFile(path+"temp.html"));
\end{lstlisting}

\subsubsection{THQt\_ShowChartsOfLineFromMatrix}\label{THQt_ShowChartsOfLineFromMatrix}

Функция возвращает строку с выводом графиков из матрицы по точкам с HTML кодами. Для добавление в html файл.


\begin{lstlisting}[label=code_syntax_THQt_ShowChartsOfLineFromMatrix,caption=Синтаксис]
template <class T> QString THQt_ShowChartsOfLineFromMatrix (T **VMHL_MatrixXY,int VMHL_N,int VMHL_M, QString TitleChart, QString NameVectorX, QString NameVectorY,QString *NameLine, bool ShowLine,bool ShowPoints,bool ShowArea,bool ShowSpecPoints);
template <class T> QString THQt_ShowChartsOfLineFromMatrix (T **VMHL_MatrixXY,int VMHL_N,int VMHL_M, QString TitleChart, QString NameVectorX, QString NameVectorY,bool ShowLine,bool ShowPoints,bool ShowArea,bool ShowSpecPoints);
template <class T> QString THQt_ShowChartsOfLineFromMatrix (T **VMHL_MatrixXY,int VMHL_N,int VMHL_M, QString TitleChart, QString NameVectorX, QString NameVectorY,QString *NameLine);
template <class T> QString THQt_ShowChartsOfLineFromMatrix (T **VMHL_MatrixXY,int VMHL_N,int VMHL_M);
\end{lstlisting}

\textbf{Входные параметры:}
 
    VMHL\_MatrixXY --- указатель на матрицу значений X и Y графиков;
 
    VMHL\_N --- количество точек;
 
    VMHL\_M --- количество столбцов матрицы (1+количество графиков);
 
    TitleChart --- заголовок графика;
 
    NameVectorX --- название оси Ox;
 
    NameVectorY --- название оси Oy;
 
    NameLine --- указатель на вектор названий графиков (для легенды) количество элементов VMHL\_M---1 (так как первый столбец --- это X значения);
 
    ShowLine --- показывать ли линию;
 
    ShowPoints --- показывать ли точки;
 
    ShowArea --- показывать ли закрашенную область под кривой;
 
    ShowSpecPoints --- показывать ли специальные точки.

\textbf{Возвращаемое значение:}

Строка с HTML кодами с выводимым графиком.

\textbf{Примечание:}

Используются случайные числа, так что рекомендуется вызвать в программе иницилизатор случайных чисел qsrand. Рекомендую так: qsrand(QDateTime::currentMSecsSinceEpoch () % 1000000);

Требует наличия в папке с html файлом файлы jsxgraph.css и jsxgraphcore.js из библиотеки JSXGraph.


\begin{lstlisting}[label=code_use_THQt_ShowChartsOfLineFromMatrix,caption=Пример использования]
QString DS=QDir::separator();
 QString path=QGuiApplication::applicationDirPath()+DS;//путь к папке

 QString Html;
 Html=HQt_BeginHtml ();

 int N=6;
 int M=2;
 double **data;
 data=new double*[N];
 for (int i=0;i<N;i++) X[i]=new double[M];
 data[0][0]=7;data[0][1]=6;
 data[1][0]=8;data[1][1]=4;
 data[2][0]=10;data[2][1]=7;
 data[3][0]=5;data[3][1]=12;
 data[4][0]=14;data[4][1]=4;
 data[5][0]=13;data[5][1]=8;

 QString *NameLine=new QString[M-1];
 NameLine[0]="Первая линия";

 Html += THQt_ShowChartsOfLineFromMatrix (data,N,M, "График", "x", "y",NameLine,true,true,true,true);

 for (int i=0;i<N;i++) delete [] data[i];
 delete [] data;
 delete [] NameLine;

 Html+=HQt_EndHtml();
 HQt_SaveFile(Html, path+"temp.html");
 ui->webView->setUrl(QUrl::fromLocalFile(path+"temp.html"));
\end{lstlisting}

\subsubsection{THQt\_ShowIndependentChartsOfLineFromMatrix}\label{THQt_ShowIndependentChartsOfLineFromMatrix}

Функция возвращает строку с выводом графиков из матрицы по точкам с HTML кодами. Для добавление в html файл.


\begin{lstlisting}[label=code_syntax_THQt_ShowIndependentChartsOfLineFromMatrix,caption=Синтаксис]
template <class T> QString THQt_ShowIndependentChartsOfLineFromMatrix (T **VMHL_MatrixXY,int *VMHL_N_EveryCol,int VMHL_M, QString TitleChart, QString NameVectorX, QString NameVectorY,QString *NameLine, bool ShowLine,bool ShowPoints,bool ShowArea,bool ShowSpecPoints);
template <class T> QString THQt_ShowIndependentChartsOfLineFromMatrix (T **VMHL_MatrixXY,int *VMHL_N_EveryCol,int VMHL_M, QString TitleChart, QString NameVectorX, QString NameVectorY,bool ShowLine,bool ShowPoints,bool ShowArea,bool ShowSpecPoints);
template <class T> QString THQt_ShowIndependentChartsOfLineFromMatrix (T **VMHL_MatrixXY,int *VMHL_N_EveryCol,int VMHL_M, QString TitleChart, QString NameVectorX, QString NameVectorY,QString *NameLine);
template <class T> QString THQt_ShowIndependentChartsOfLineFromMatrix (T **VMHL_MatrixXY,int *VMHL_N_EveryCol,int VMHL_M);
\end{lstlisting}

\textbf{Входные параметры:}
 
VMHL\_MatrixXY --- указатель на матрицу значений X и Н графиков;
 
VMHL\_N\_EveryCol --- количество элементов в каждом столбце (так как столбцы идут по парам, то число элементов в нечетном и
 
следующем за ним четном столбце должны совпадать, например 10,10,5,5,7,7);
 
VMHL\_M --- количество столбцов матрицы (должно быть четным числом конечно);
 
TitleChart --- заголовок графика;
 
NameVectorX --- название оси Ox;
 
NameVectorY --- название оси Oy;
 
NameLine --- указатель на вектор названий графиков (для легенды) количество элементов VMHL\_M/2;
 
ShowLine --- показывать ли линию;
 
ShowPoints --- показывать ли точки;
 
ShowArea --- показывать ли закрашенную область под кривой;
 
ShowSpecPoints --- показывать ли специальные точки.

\textbf{Возвращаемое значение:}

Строка с HTML кодами с выводимым графиком.

\textbf{Примечание:}

Используются случайные числа, так что рекомендуется вызвать в программе иницилизатор случайных чисел qsrand. Рекомендую так: qsrand(QDateTime::currentMSecsSinceEpoch () % 1000000);

Требует наличия в папке с html файлом файлы jsxgraph.css и jsxgraphcore.js из библиотеки JSXGraph.

Нечетные столбцы - это значения координат X графиков. Следующие за ними четные столбцы - соответствующие значения Y. То есть графики друг от друга независимы.


\subsubsection{THQt\_ShowTwoChartsOfLine}\label{THQt_ShowTwoChartsOfLine}

Функция возвращает строку с выводом некоторого двух графиков по точкам с HTML кодами. Для добавление в html файл. У обоих графиков одинаковый массив значений X.


\begin{lstlisting}[label=code_syntax_THQt_ShowTwoChartsOfLine,caption=Синтаксис]
template <class T> QString THQt_ShowTwoChartsOfLine (T *VMHL_VectorX,T *VMHL_VectorY1,T *VMHL_VectorY2, int VMHL_N, QString TitleChart, QString NameVectorX, QString NameVectorY,QString NameLine1, QString NameLine2,bool ShowLine,bool ShowPoints,bool ShowArea,bool ShowSpecPoints);
template <class T> QString THQt_ShowTwoChartsOfLine (T *VMHL_VectorX,T *VMHL_VectorY1,T *VMHL_VectorY2, int VMHL_N, QString TitleChart, QString NameVectorX, QString NameVectorY,bool ShowLine,bool ShowPoints,bool ShowArea,bool ShowSpecPoints);
template <class T> QString THQt_ShowTwoChartsOfLine (T *VMHL_VectorX,T *VMHL_VectorY1,T *VMHL_VectorY2, int VMHL_N, QString TitleChart, QString NameVectorX, QString NameVectorY,QString NameLine1, QString NameLine2);
template <class T> QString THQt_ShowTwoChartsOfLine (T *VMHL_VectorX,T *VMHL_VectorY1,T *VMHL_VectorY2, int VMHL_N);
\end{lstlisting}

\textbf{Входные параметры:}
 
    VMHL\_VectorX --- указатель на вектор координат X точек;
 
    VMHL\_VectorY1 --- указатель на вектор координат Y точек первой линии;
 
    VMHL\_VectorY2 --- указатель на вектор координат Y точек второй линии;
 
    VMHL\_N --- количество точек;
 
    TitleChart --- заголовок графика;
 
    NameVectorX --- название оси Ox;
 
    NameVectorY --- название оси Oy;
 
    NameLine1 --- название первого графика (для легенды);
 
    NameLine2 --- название второго графика (для легенды);
 
    ShowLine --- показывать ли линию;
 
    ShowPoints --- показывать ли точки;
 
    ShowArea --- показывать ли закрашенную область под кривой;
 
    ShowSpecPoints --- показывать ли специальные точки.

\textbf{Возвращаемое значение:}

Строка с HTML кодами с выводимым графиком.

\textbf{Примечание:}

Используются случайные числа, так что рекомендуется вызвать в программе иницилизатор случайных чисел qsrand. Рекомендую так: qsrand(QDateTime::currentMSecsSinceEpoch () % 1000000);

Требует наличия в папке с html файлом файлы jsxgraph.css и jsxgraphcore.js из библиотеки JSXGraph.


\begin{lstlisting}[label=code_use_THQt_ShowTwoChartsOfLine,caption=Пример использования]
QString DS=QDir::separator();
QString path=QGuiApplication::applicationDirPath()+DS;//путь к папке

QString Html;
Html=HQt_BeginHtml ();

int N=6;
double *dataX=new double [N];
double *dataY1=new double [N];
double *dataY2=new double [N];
dataX[0]=7;dataY1[0]=6;dataY2[0]=1;
dataX[1]=8;dataY1[1]=4;dataY2[0]=2;
dataX[2]=10;dataY1[2]=7;dataY2[0]=3;
dataX[3]=5;dataY1[3]=12;dataY2[0]=4;
dataX[4]=14;dataY1[4]=4;dataY2[0]=4;
dataX[5]=13;dataY1[5]=8;dataY2[0]=3;

Html += THQt_ShowTwoChartsOfLine (dataX,dataY1,dataY2,N,"Тестовый график","x","y","количество деревьев","количество домов",true,true,true,true);

delete []dataX;
delete []dataY1;
delete []dataY2;

Html+=HQt_EndHtml();
HQt_SaveFile(Html, path+"temp.html");
ui->webView->setUrl(QUrl::fromLocalFile(path+"temp.html"));
\end{lstlisting}

\subsubsection{THQt\_ShowTwoIndependentChartsOfLine}\label{THQt_ShowTwoIndependentChartsOfLine}

Функция возвращает строку с выводом некоторого двух независимых графиков по точкам с HTML кодами. Для добавление в html файл.


\begin{lstlisting}[label=code_syntax_THQt_ShowTwoIndependentChartsOfLine,caption=Синтаксис]
template <class T> QString THQt_ShowTwoIndependentChartsOfLine (T *VMHL_VectorX1,T *VMHL_VectorY1,int VMHL_N1,T *VMHL_VectorX2,T *VMHL_VectorY2, int VMHL_N2, QString TitleChart, QString NameVectorX, QString NameVectorY,QString NameLine1, QString NameLine2,bool ShowLine,bool ShowPoints,bool ShowArea,bool ShowSpecPoints);
template <class T> QString THQt_ShowTwoIndependentChartsOfLine (T *VMHL_VectorX1,T *VMHL_VectorY1,int VMHL_N1,T *VMHL_VectorX2,T *VMHL_VectorY2, int VMHL_N2, QString TitleChart, QString NameVectorX, QString NameVectorY,bool ShowLine,bool ShowPoints,bool ShowArea,bool ShowSpecPoints);
template <class T> QString THQt_ShowTwoIndependentChartsOfLine (T *VMHL_VectorX1,T *VMHL_VectorY1,int VMHL_N1,T *VMHL_VectorX2,T *VMHL_VectorY2, int VMHL_N2, QString TitleChart, QString NameVectorX, QString NameVectorY,QString NameLine1, QString NameLine2);
template <class T> QString THQt_ShowTwoIndependentChartsOfLine (T *VMHL_VectorX1,T *VMHL_VectorY1,int VMHL_N1,T *VMHL_VectorX2,T *VMHL_VectorY2, int VMHL_N2);
\end{lstlisting}

\textbf{Входные параметры:}
 
    VMHL\_VectorX1 --- указатель на вектор координат X точек первой линии;
 
    VMHL\_VectorY1 --- указатель на вектор координат Y точек первой линии;
 
    VMHL\_N1 --- количество точек первой линии;
 
    VMHL\_VectorX2 --- указатель на вектор координат X точек второй линии;
 
    VMHL\_VectorY2 --- указатель на вектор координат Y точек второй линии;
 
    VMHL\_N2 --- количество точек второй линии;
 
    TitleChart --- заголовок графика;
 
    NameVectorX --- название оси Ox;
 
    NameVectorY --- название оси Oy;
 
    NameLine1 --- название первого графика (для легенды);
 
    NameLine2 --- название второго графика (для легенды);
 
    ShowLine --- показывать ли линию;
 
    ShowPoints --- показывать ли точки;
 
    ShowArea --- показывать ли закрашенную область под кривой;
 
    ShowSpecPoints --- показывать ли специальные точки.

\textbf{Возвращаемое значение:}

Строка с HTML кодами с выводимым графиком.

\textbf{Примечание:}

Используются случайные числа, так что рекомендуется вызвать в программе иницилизатор случайных чисел qsrand. Рекомендую так: qsrand(QDateTime::currentMSecsSinceEpoch () % 1000000);

Требует наличия в папке с html файлом файлы jsxgraph.css и jsxgraphcore.js из библиотеки JSXGraph.


\begin{lstlisting}[label=code_use_THQt_ShowTwoIndependentChartsOfLine,caption=Пример использования]
QString DS=QDir::separator();
QString path=QGuiApplication::applicationDirPath()+DS;//путь к папке

QString Html;
Html=HQt_BeginHtml ();

int N1=6;
double *dataX1=new double [N1];
double *dataY1=new double [N1];
dataX1[0]=7;dataY1[0]=6;
dataX1[1]=8;dataY1[1]=4;
dataX1[2]=10;dataY1[2]=7;
dataX1[3]=5;dataY1[3]=12;
dataX1[4]=14;dataY1[4]=4;
dataX1[5]=13;dataY1[5]=8;

int N2=3;
double *dataX2=new double [N1];
double *dataY2=new double [N2];
dataX2[0]=1;dataY2[0]=8;
dataX2[1]=2;dataY2[1]=4;
dataX2[2]=3;dataY2[2]=5;

Html += THQt_ShowTwoIndependentChartsOfLine (dataX1,dataY1,N1,dataX2,dataY2,N2,"Тестовый график","x","y","количество деревьев","количество домов",true,true,true,true);

delete []dataX1;
delete []dataY1;
delete []dataX2;
delete []dataY2;

Html+=HQt_EndHtml();
HQt_SaveFile(Html, path+"temp.html");
ui->webView->setUrl(QUrl::fromLocalFile(path+"temp.html"));
\end{lstlisting}

\subsubsection{THQt\_ShowTwoIndependentChartsOfPointsAndLine}\label{THQt_ShowTwoIndependentChartsOfPointsAndLine}

Функция возвращает строку с выводом некоторого двух независимых графиков по точкам с HTML кодами. Для добавление в html файл. Один график выводится в виде точек, а второй в виде линии. Удобно для отображения регрессий. У обоих графиков разные массивы значений X и Y.


\begin{lstlisting}[label=code_syntax_THQt_ShowTwoIndependentChartsOfPointsAndLine,caption=Синтаксис]
template <class T> QString THQt_ShowTwoIndependentChartsOfPointsAndLine (T *VMHL_VectorX1,T *VMHL_VectorY1,int VMHL_N1,T *VMHL_VectorX2,T *VMHL_VectorY2, int VMHL_N2, QString TitleChart, QString NameVectorX, QString NameVectorY,QString NameLine1, QString NameLine2,bool ShowLine,bool ShowPoints,bool ShowArea,bool ShowSpecPoints);
template <class T> QString THQt_ShowTwoIndependentChartsOfPointsAndLine (T *VMHL_VectorX1,T *VMHL_VectorY1,int VMHL_N1,T *VMHL_VectorX2,T *VMHL_VectorY2, int VMHL_N2, QString TitleChart, QString NameVectorX, QString NameVectorY,bool ShowLine,bool ShowPoints,bool ShowArea,bool ShowSpecPoints);
template <class T> QString THQt_ShowTwoIndependentChartsOfPointsAndLine (T *VMHL_VectorX1,T *VMHL_VectorY1,int VMHL_N1,T *VMHL_VectorX2,T *VMHL_VectorY2, int VMHL_N2, QString TitleChart, QString NameVectorX, QString NameVectorY,QString NameLine1, QString NameLine2);
template <class T> QString THQt_ShowTwoIndependentChartsOfPointsAndLine (T *VMHL_VectorX1,T *VMHL_VectorY1,int VMHL_N1,T *VMHL_VectorX2,T *VMHL_VectorY2, int VMHL_N2);
\end{lstlisting}

\textbf{Входные параметры:}
 
    VMHL\_VectorX1 --- указатель на вектор координат X точек первой линии;
 
    VMHL\_VectorY1 --- указатель на вектор координат Y точек первой линии;
 
    VMHL\_N1 --- количество точек первой линии;
 
    VMHL\_VectorX2 --- указатель на вектор координат X точек второй линии;
 
    VMHL\_VectorY2 --- указатель на вектор координат Y точек второй линии;
 
    VMHL\_N2 --- количество точек второй линии;
 
    TitleChart --- заголовок графика;
 
    NameVectorX --- название оси Ox;
 
    NameVectorY --- название оси Oy;
 
    NameLine1 --- название первого графика (для легенды);
 
    NameLine2 --- название второго графика (для легенды);
 
    ShowLine --- показывать ли линию;
 
    ShowPoints --- показывать ли точки;
 
    ShowArea --- показывать ли закрашенную область под кривой;
 
    ShowSpecPoints --- показывать ли специальные точки.

\textbf{Возвращаемое значение:}

Строка с HTML кодами с выводимым графиком.

\textbf{Примечание:}

Используются случайные числа, так что рекомендуется вызвать в программе иницилизатор случайных чисел qsrand. Рекомендую так: qsrand(QDateTime::currentMSecsSinceEpoch () % 1000000);

Требует наличия в папке с html файлом файлы jsxgraph.css и jsxgraphcore.js из библиотеки JSXGraph.


\subsection{Показ математических выражений}

\subsubsection{THQt\_ShowMatrix}\label{THQt_ShowMatrix}

Функция возвращает строку с выводом некоторой матрицы VMHL\_Matrix с HTML кодами. Для добавление в html файл.


\begin{lstlisting}[label=code_syntax_THQt_ShowMatrix,caption=Синтаксис]
template <class T> QString THQt_ShowMatrix (T *VMHL_Matrix, int VMHL_N, int VMHL_M, QString TitleMatrix, QString NameMatrix);
template <class T> QString THQt_ShowMatrix (T *VMHL_Matrix, int VMHL_N, int VMHL_M, QString NameMatrix);
template <class T> QString THQt_ShowMatrix (T *VMHL_Matrix, int VMHL_N, int VMHL_M);
\end{lstlisting}

\textbf{Входные параметры:}
 
    VMHL\_Matrix --- указатель на выводимую матрицу;
 
    VMHL\_N --- количество строк в матрице;
 
    VMHL\_M --- количество столбцов в матрице;
 
    TitleMatrix --- заголовок выводимой матрицы;
 
    NameMatrix --- обозначение матрицы.

\textbf{Возвращаемое значение:}

Строка с HTML кодами с выводимой матрицей.


\subsubsection{THQt\_ShowMatrix2}\label{THQt_ShowMatrix2}

Функция возвращает строку с выводом некоторой матрицы VMHL\_Matrix с HTML кодами. Для добавление в html файл. В качестве матрицы выступает массив QStringList, где количество QStringList - это количество строк. Каждый QStringList --- это одна строка.


\begin{lstlisting}[label=code_syntax_THQt_ShowMatrix2,caption=Синтаксис]
QString THQt_ShowMatrix (QStringList *VMHL_Matrix, int VMHL_N, QString TitleMatrix, QString NameMatrix);
QString THQt_ShowMatrix (QStringList *VMHL_Matrix, int VMHL_N, QString NameMatrix);
QString THQt_ShowMatrix (QStringList *VMHL_Matrix, int VMHL_N);
\end{lstlisting}

\textbf{Входные параметры:}

    VMHL\_Matrix --- указатель на выводимую матрицу;
 
    VMHL\_N --- количество строк в матрице;
 
    TitleMatrix --- заголовок выводимой матрицы;
 
    NameMatrix --- обозначение матрицы.

\textbf{Возвращаемое значение:}

Строка с HTML кодами с выводимой матрицей.


\subsubsection{THQt\_ShowNumber}\label{THQt_ShowNumber}

Функция возвращает строку с выводом некоторого числа VMHL\_X с HTML кодами. Для добавление в html файл.


\begin{lstlisting}[label=code_syntax_THQt_ShowNumber,caption=Синтаксис]
template <class T> QString THQt_ShowNumber (T VMHL_X, QString TitleX, QString NameX);
template <class T> QString THQt_ShowNumber (T VMHL_X, QString NameX);
template <class T> QString THQt_ShowNumber (T VMHL_X);
\end{lstlisting}

\textbf{Входные параметры:}

VMHL\_X --- выводимое число;
 
    TitleX --- заголовок выводимого числа;
 
    NameX --- обозначение числа.

\textbf{Возвращаемое значение:}

Строка с HTML кодами с выводимым числом.


\subsubsection{THQt\_ShowVector}\label{THQt_ShowVector}

Функция возвращает строку с выводом некоторого вектора VMHL\_Vector с HTML кодами. Для добавление в html файл.


\begin{lstlisting}[label=code_syntax_THQt_ShowVector,caption=Синтаксис]
template <class T> QString THQt_ShowVector (T *VMHL_Vector, int VMHL_N, QString TitleVector, QString NameVector);
template <class T> QString THQt_ShowVector (T *VMHL_Vector, int VMHL_N, QString NameVector);
template <class T> QString THQt_ShowVector (T *VMHL_Vector, int VMHL_N);
\end{lstlisting}

\textbf{Входные параметры:}

VMHL\_Vector --- указатель на выводимый вектор;
 
    VMHL\_N --- количество элементов вектора;
 
    TitleVector --- заголовок выводимого вектора;
 
    NameVector --- обозначение вектора.

\textbf{Возвращаемое значение:}

Строка с HTML кодами с выводимым вектором.


\subsubsection{THQt\_ShowVector2}\label{THQt_ShowVector2}

Функция возвращает строку с выводом некоторого списка строк VMHL\_Vector с HTML кодами. Для добавление в html файл.


\begin{lstlisting}[label=code_syntax_THQt_ShowVector2,caption=Синтаксис]
QString THQt_ShowVector (QStringList VMHL_Vector, QString TitleVector, QString NameVector);
QString THQt_ShowVector (QStringList VMHL_Vector, QString NameVector);
QString THQt_ShowVector (QStringList VMHL_Vector);
\end{lstlisting}

\textbf{Входные параметры:}
 
VMHL\_Vector --- указатель на список строк QStringList;
 
    VMHL\_N --- количество элементов вектора;
 
    TitleVector --- заголовок выводимого вектора;
 
    NameVector --- обозначение вектора.

\textbf{Возвращаемое значение:}

Строка с HTML кодами с выводимым вектором.


\subsubsection{THQt\_ShowVectorT}\label{THQt_ShowVectorT}

Функция возвращает строку с выводом некоторого вектора VMHL\_Vector в траснпонированном виде с HTML кодами. Для добавление в html файл.


\begin{lstlisting}[label=code_syntax_THQt_ShowVectorT,caption=Синтаксис]
template <class T> QString THQt_ShowVectorT (T *VMHL_Vector, int VMHL_N, QString TitleVector, QString NameVector);
template <class T> QString THQt_ShowVectorT (T *VMHL_Vector, int VMHL_N, QString NameVector);
template <class T> QString THQt_ShowVectorT (T *VMHL_Vector, int VMHL_N);
\end{lstlisting}

\textbf{Входные параметры:}
 
    VMHL\_Vector --- указатель на выводимый вектор;
 
    VMHL\_N --- количество элементов вектора;
 
    TitleVector --- заголовок выводимого вектора;
 
    NameVector --- обозначение вектора.

\textbf{Возвращаемое значение:}

 
    Строка с HTML кодами с выводимым вектором.


\subsection{Показ текста}

\subsubsection{HQt\_ShowAlert}\label{HQt_ShowAlert}

Функция возвращает строку с выводом некоторого предупреждения. Для добавление в html файл.


\begin{lstlisting}[label=code_syntax_HQt_ShowAlert,caption=Синтаксис]
QString HQt_ShowAlert (QString String);
\end{lstlisting}

\textbf{Входные параметры:}

String --- непосредственно выводимая строка.

\textbf{Возвращаемое значение:}

Строка с HTML кодами с выводимым предупреждением.


\subsubsection{HQt\_ShowH1}\label{HQt_ShowH1}

Функция возвращает строку с выводом некоторой строки в виде заголовка. Для добавление в html файл.


\begin{lstlisting}[label=code_syntax_HQt_ShowH1,caption=Синтаксис]
QString HQt_ShowH1 (QString String);
\end{lstlisting}

\textbf{Входные параметры:}

String --- непосредственно выводимая строка.

\textbf{Возвращаемое значение:}

Строка с HTML кодами с выводимой строкой.


\subsubsection{HQt\_ShowHr}\label{HQt_ShowHr}

Функция возвращает строку с выводом горизонтальной линии. Для добавление в html файл.


\begin{lstlisting}[label=code_syntax_HQt_ShowHr,caption=Синтаксис]
QString HQt_ShowHr ();
\end{lstlisting}

\textbf{Входные параметры:}

Отсутствуют.

\textbf{Возвращаемое значение:}

Строка с HTML кодами с выводимой строкой.


\subsubsection{HQt\_ShowSimpleText}\label{HQt_ShowSimpleText}

Функция возвращает строку с выводом некоторой строки с HTML кодами без всякого излишества. Для добавление в html файл.


\begin{lstlisting}[label=code_syntax_HQt_ShowSimpleText,caption=Синтаксис]
QString HQt_ShowSimpleText (QString String);
\end{lstlisting}

\textbf{Входные параметры:}

String --- непосредственно выводимая строка.

\textbf{Возвращаемое значение:}

Строка с HTML кодами с выводимой строкой.


\subsubsection{HQt\_ShowText}\label{HQt_ShowText}

Функция возвращает строку с выводом некоторой строки с HTML кодами. Для добавление в html файл.


\begin{lstlisting}[label=code_syntax_HQt_ShowText,caption=Синтаксис]
QString HQt_ShowText (QString TitleX);
\end{lstlisting}

\textbf{Входные параметры:}

TitleX --- непосредственно выводимая строка.

\textbf{Возвращаемое значение:}

Строка с HTML кодами с выводимой строкой.


\subsubsection{THQt\_NumberToText}\label{THQt_NumberToText}

Функция выводит число VMHL\_X в строку.


\begin{lstlisting}[label=code_syntax_THQt_NumberToText,caption=Синтаксис]
template <class T> QString THQt_NumberToText (T VMHL_X);
\end{lstlisting}

\textbf{Входные параметры:}

VMHL\_X --- выводимое число.

\textbf{Возвращаемое значение:}

Строка, в которой записано число.

\end{document}