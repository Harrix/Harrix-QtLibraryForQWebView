\textbf{Входные параметры:}
 
    VMHL\_VectorX --- указатель на вектор координат X точек;
 
    VMHL\_VectorY1 --- указатель на вектор координат Y точек первой линии;
 
    VMHL\_VectorY2 --- указатель на вектор координат Y точек второй линии;
 
    VMHL\_N --- количество точек;
 
    TitleChart --- заголовок графика;
 
    NameVectorX --- название оси Ox;
 
    NameVectorY --- название оси Oy;
 
    NameLine1 --- название первого графика (для легенды);
 
    NameLine2 --- название второго графика (для легенды);
 
    ShowLine --- показывать ли линию;
 
    ShowPoints --- показывать ли точки;
 
    ShowArea --- показывать ли закрашенную область под кривой;
 
    ShowSpecPoints --- показывать ли специальные точки.

\textbf{Возвращаемое значение:}

Строка с HTML кодами с выводимым графиком.

\textbf{Примечание:}

Используются случайные числа, так что рекомендуется вызвать в программе иницилизатор случайных чисел qsrand. Рекомендую так: qsrand(QDateTime::currentMSecsSinceEpoch () % 1000000);

Требует наличия в папке с html файлом файлы jsxgraph.css и jsxgraphcore.js из библиотеки JSXGraph.